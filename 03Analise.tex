% Prof. Dr. Ausberto S. Castro Vera
% UENF - CCT - LCMAT - Curso de Ci\^{e}ncia da Computa\c{c}\~{a}o
% Campos, RJ,  2015
% Disciplina: An\'{a}lise e Projeto de Sistemas
% Aluno:


\chapter{Etapa de An\'{a}lise}
Os requisitos são caracteristicas, atributos e habilidade que um sistema deve conter para que seja um software
para aquele determinado serviço ou produto.
 \section{Requisitos do sistema}
 \subsection{Requisitos}
  \begin{itemize}
      \item Aluno
      \item Professor
      \item Serviço Online
      \item Acesso ao Sistema
      \item Segurança do Servidor
      \item Disponibilidade do Sistema
  \end{itemize}
  
  \subsection{Definição}
  \begin{itemize}
      \item Os alunos deverão fazer matrículas, cadastrar materias, visualizar notas e excluir materias
      \item Os professores deverão criar turma, adicionar aluno, excluir aluno, finalizar turma e inserir nota
      \item O serviço online deve automatizar o cadastro de matrícula e outros serviços academicos.
      \item Acesso ao Sistema deve conter o número de matrícula e a senha personalizada pelo cliente com mais de 8 caracteres.	
      \item Segurança do Servidor deve ter uma proteção para que não burle o sistema ou danifiquem pessoalmente. 
      \item Disponibilidade do Sistema  deve ser 24 horas por dia, 7 dias por semana.
  \end{itemize}
   
 \subsection{Especificação dos Requisitos}
   \begin{itemize}
    \item Alunos
      \begin{itemize}
       \item O aluno passou no vestibular.
       \item Eles estão aptos a usar o sistemas
       \item O aluno tem matrícula e senha personalizada
       \item O aluno terá a disposição um computador caso não tenha computador.
      \end{itemize}

     \item Professores
     \begin{itemize}
      \item O professor é concursado ou contratado.
      \item O professor tem acesso as matérias
      \item O professor pode criar turmas.
      \item O professor tem acesso diferenciado
     \end{itemize}
     
     \item Serviço Online
     \begin{itemize}
      \item O serviço tem que ser armazenado em um servidor
      \item O serviço tem que utilizar tecnologias web
      \item O serviço tem que usar uma tecnologia escalavel
      \item O serviço tem que usar as melhores práticas para um bom desempenho
      \item O serviço tem que ser adaptável.
     \end{itemize}
     
     \item Acesso ao Sistema
     \begin{itemize}
      \item O acesso tem que ser diferente para professores e alunos.
      \item O acesso só será permitido para alunos ou professores cadastrados.
      \item O acesso deverá ser feito pela matrícula do aluno ou professor e uma senha personalizada.
      \item O acesso permitirá visualizar conteúdos distintos.
     \end{itemize}
     
     \item Segurança do Servidor
      \begin{itemize}
       \item O servidor ficará em um local com temperatura baixa
       \item O servidor terá duas portas de segurança
       \item O servidor terá hardware de alta perfomace
       \item O servidor será criptografado.
      \end{itemize}

     
     \item Disponibilidade do sistema
        \begin{itemize}
         \item O sistema conterá duas redes de internet disponível.
         \item O sistema funcionará mesmo offiline, armazendo blocos de texto e quando tiver conexão será enviado ao servidor.
        \end{itemize}


   \end{itemize}

   \subsection{Requisitos de Rede}
     \begin{itemize}
      \item O sistema realizará a automação dos processos seletivos academicos e permitirá que todo meio academico possa está sintonizado a suas notas e matrícuas.
      \item Deverá ser usada a programação Orientada a Objeto, pois atenderá melhor ao sistema.
      \item O usuário terá uma interface gráfica amigável para poder acessar o sistema.
      \item O sistema conterá diversas linguas, pois a universidade recebe alunos extrangeiros.
      \item A linguagem a ser utilizada deve ser portável.
     \end{itemize}
     
     \subsection{Requisitos de Subsistema}
      \begin{itemize}
       \item  O sistema deve ficar disponível 24 horas por dias, 7 dias por semana.
       \item  O sistema deverá utilizar dos melhores switchs, roteadores e cabos de redes de melhor qualidade, afim de não haver desgaste ou danos aos componetes.
       \item  A transmissão de dados deverá ser de 100mb/s para que o servidor não fique lento.
       \item O Sistema terá um sistema de backup do tipo RAD 10.
      \end{itemize}

     \subsection{Definições}
       \begin{itemize}
        \item Funcionais
           \begin{itemize}
            \item O sistema deverá conter computadores robustos, pois armazenará todo os dados em um cloud, próprio.
            \item O sistema deverá se comunicar com eficiência.
            \item O sistema deverá fazer o cadastro de todas a comunidade academica.
            \item O servidor deverá monitorar e enviar emails para o administrado em caso de falha.
           \end{itemize}
           
         \item Não funcionais
           \begin{itemize}
            \item O sistema utilizará o sistema operacional Linux.
            \item Manutenção preventiva.
            \item Um sistema com acesso simultâneo.
            \item Sistema com designer inovador.
            \item Cada usuário terá o seu tipo de acesso.
           \end{itemize}

        \item Características não desejáveis
	 \begin{itemize}
	  \item Instabilidade do sistema
	  \item Os meio academico não utilizar o serviço.
	  \item Baixa qualidade do sistema.
	  \item Sistema indisponível.
	 \end{itemize}
	 
	 \item Do produto
	 \begin{itemize}
	  \item Resultado satisfatório
	  \item Automatização das funções
	  \item Diminuição do uso do papel
	  \item Confiança na utilização
	  \item Satisfação do meio academico.
	 \end{itemize} 
	 
	 \item Da organização
	   \begin{itemize}
	    \item Extinção do papel.
	    \item Rapidez no serviço.
	    \item Automatização das tarefas.
	    \item Eficiência do serviço.
	   \end{itemize} 
	   
	 \item Segurança
	   \begin{itemize}
	    \item Sigilo dos dados
	    \item Criptografia das senhas
	    \item Acesso permitido a usuário cadastrado.
	    \item Estabilidade do site
	   \end{itemize}
	   
       \end{itemize}
