% Prof. Dr. Ausberto S. Castro Vera
% UENF - CCT - LCMAT - Curso de Ci\^{e}ncia da Computa\c{c}\~{a}o
% Campos, RJ,  2015
% Disciplina: An\'{a}lise e Projeto de Sistemas
% Aluno: Rodolfo Gomes Peixoto



\chapter{ Introdu\c{c}\~{a}o }


   \section{Descri\c{c}\~{a}o do Sistema Computacional a desenvolver}

     O SIAUENF é um sistema ERP de gestão acadêmico. O software incorporará todos os processos do meio universitário, desde o ingresso do aluno até sua saída. O sistema será dividido em módulos que segue abaixo:

     Vestibular - ( Conterá todas as notas obtida pelo aluno no vestibular, classificação, alocação do aluno nas salas para fazer a prova )

Acadêmico - ( Área onde o aluno poderá consultar: notas, provas, matérias, inclusão, exclusão, horas cursadas, chat direto com o professor, coeficiente de rendimento e o coeficiente de rendimento mensal , transferência, histórico escolar, diploma, cancelamento, será dividido para mestrado, graduando, pós, doutorado.)

Financeiro - ( Cadastro de bolsas, administração da verba, fluxo de caixa, divisão de verba, contas a pagar, contas a receber, cobranças bancárias, dentre outras.)




   \section{Identificando as componentes do meu sistema}
   

     
     
    \textbf{Hardware}
     \begin{itemize}
         \item Processador: i7
         \item HD: 1Terabyte
         \item RAM: 16 GB
         \item Placa Mãe: ASUS
        \item Roteador
       \item Cabos RJ45
   \end{itemize}
     
   \textbf{Software}
   \begin{itemize}
    \item Sistema Operacional: Ubuntu
    \item NGIX
    \item Ruby on Rails
    \item Terminal
    \item Sublime Text 3
    \item Chrome
    \item Firefox
   \end{itemize}
   
   \textbf{Banco de Dados}
   \begin{itemize}
    \item MySQL
    \item MongoDB
   \end{itemize}

    
 


 