% Prof. Dr. Ausberto S. Castro Vera
% UENF - CCT - LCMAT - Curso de Ci\^{e}ncia da Computa\c{c}\~{a}o
% Campos, RJ,  2015
% Disciplina: An\'{a}lise e Projeto de Sistemas
% Aluno: Rodolfo Gomes Peixoto


\chapter{Etapa de An\'{a}lise}
Os requisitos são caracteristicas, atributos e habilidade que um sistema deve conter para que seja um software
para aquele determinado serviço ou produto.
 \section{Requisitos do sistema}
 \subsection{Requisitos}
  \begin{itemize}
      \item Servidor
      \item Computadores
      \item Cabos de rede
      \item Switchs
      \item No-break
      \item Sistema Operacional
      \item Orientação a Objeto
      \item Interface Gráfica
      \item Página Web
      \item Leitor de PDF
      \item Cadastro de Alunos
      \item Cadastro de Professores
      \item Cadastro de Matérias
      \item Cadastro de Notas
      \item Exclusão de matéria
      \item Manual do Sistema
      \item Manual do Sistema Operacional
      \item Relatórios
      \item Documentação Digital
      \item Documento genérico para assuntos gerais
      \item Registrar Log para cada usuário
      \item Backup do Sistema
      \item Treinamento dos Alunos
      \item Treinamento dos Professores
      \item Treinamento dos Funcionários
      \item Treinamento Horário Comercial
      \item Treinamento Online
      \item Manutenção do treinamento
      \item Implementação da Rede
      \item Serviço Online
      \item Segurança do Servidor
      \item Disponibilidade do Sistema
  \end{itemize}
  
  \subsection{Definição}
  \begin{itemize}
\item O sistema deve ter um servidor onde esta o aplicativo.
\item A rede do sistema deve  funcionar usando cabeamento.
\item Os computadores estarão disponíveis para o acesso e manutenção dos mesmos.
\item O sistema deve utilizar roteadores para a comunicação externa.
\item Cada computador deve estar associado a um sistema operacional.
\item o sistema deve conter uma interface gráfica para que facilite a interação usuário-máquina.
\item Deve utilizar uma linguagem orientada a objetos para facilitar o desenvolvimento para uma linguagem mais natural.
\item O sistema deve ter um operador de sistema para fazer a manutenção e acesso.
\item Deve ter para poder verificar algum erro que possa estar ocorrendo na conexão.
\item É necessário para que o serviço de atendimento online esteja em operação.
\item É importante para saber a quantidade de alunos
\item É importante para fazer o controle dos professores que terão acesso ao sistema
\item O sistema deve conter as notas dos alunos para maior facilidade
\item Terá as datas das provas de todas as disciplinas, através do portal do aluno.
\item O acesso ao sistema deverá ser feito utilizando nome do usuário e senha.
\item Deve ser executado o treinamento de todo o pessoal do sistema.
\item A Segurança do Site garante sigilo dos dados do cliente.
\item O Controle de Serviço é feito online de acordo com a disponibilidade do funcionário para que as chamadas sejam atendidas.
\item O sistema deverá ter o relatório de todos os dados do sistema.
\item No sistema os alunos poderão fazer a inclusão de disciplinas.
\item Os alunos também terão a opção de fazer a exclusão das disciplinas
\item O no-break deve impedir o desligamento das máquinas de forma brusca para não ocorrer possíveis danos aos computadores.
\item Permitir a leitura de arquivos em formatos de PDF.
\item Não permite oscilações de energia nos computadores.
\item Salvar as informações do sistema periodicamente.
\item A pré-matricula estará disponível para maior facilidade.
\item o trancamento de disciplinas pode ser feito caso o aluno desista da disciplina.
\item O aluno pode fazer o trancamento da matrícula.
\item O aluno terá a opção de fazer o cancelamento da matrícula.
\item Deve permitir o envio de requerimentos ao setor responsável.
  \end{itemize}
   
 \subsection{Especificação dos Requisitos}
   \begin{itemize}\item Professores
	\begin{itemize}
	\item Estar apto a dar aula.
	\item Estar apto a trabalhar com computadores.
	\item Estar disponível para reuniões.
	\item Seguir regras estabelecidas pelo sistema.
	\end{itemize}
\item Alunos
	\begin{itemize}
	\item Estar matriculados
	\item Estar com a documentação em dia com a secretaria
	\item Estar em dia com a biblioteca
	\end{itemize}
\item Rede de Computadores
	\begin{itemize}
	\item Serviço Online
	\item É feito através da Internet
	\item O usuário acessa o site da empresa
	\item É oferecido acesso rápido a informações sobre o atendimento
	\item Ingressar informações de acesso (usuário, senha) 
	\item Verificar usuário cadastrado 
	\item Verificar senha cadastrada 
	\item Liberar perfil de usuário 
	\item Liberar área e ferramentas de trabalho
	\item Clicar no ícone de chamadas
	\item Confirmar dados pessoais e endereço
	\item Acessar pagamento via boleto bancário ou cartão de crédito
	\item Confirmação do pagamento
	\item Enviar chamada
	\end{itemize}
\item Acesso a internet
	\begin{itemize}
	\item Ligar o computador
	\item O funcionário acessa o site da empresa
	\item Verificar acesso à Internet
	\item Verificar funcionário cadastrado 
	\item Verificar senha cadastrada 
	\item Liberar perfil de funcionário 
	\item Liberar área e ferramentas de trabalho
	\end{itemize}
\item Serviço de Tele Atendimento
	\begin{itemize}
	\item O usuário executar a solicitação do serviço através do seu próprio telefone
	\item Informa os seus dados pessoais
	\item O usuário pode ligar de novo e pedir o cancelamento da sua solicitação
	\item O usuário pode ligar para dar sua opinião para a melhoria do atendimento
	\end{itemize}
\item Acesso ao Sistema
	\begin{itemize}
	\item Abrir janela inicial com interface de acesso 
	\item Ingressar informações de acesso (usuário, senha) 
	\item Verificar aluno cadastrado 
	\item Verificar senha cadastrada 
	\item Liberar perfil do aluno 
	\item Liberar área e ferramentas de trabalho 
	\item Permitir acesso à Internet 
	\item Ver relação de chamadas
	\item Finalizar processo
	\end{itemize}
\item Segurança do Servidor
	\begin{itemize}
	\item Instalar os últimos patches do fabricante
	\item Instalar ferramentas de detecção de invasão
	\item O servidor armazena arquivos de diversos usuários
	\item O usuário deve escolher senhas fortes
	\item O sistema é configurado para só aceitar senhas que tenham, por exemplo, um número de caracteres maior do que 8, que sejam por símbolos especiais, dígitos numéricos e caracteres alfabéticos, etc.
	\item Manter o sistema operacional atualizado
	\item Configurar e manter logs atualizados
	\end{itemize}
\item Segurança do Servidor
	\begin{itemize}
	\item Realizar ajustes de defesa em tempo real
	\item Instalar ferramentas de detecção de invasão
	\item Configurar e manter logs atualizados
	\item Instalar aplicativos que garante que os dados armazenados estejam seguros
	\end{itemize}
\item Controle do Serviço
	\begin{itemize}
	\item Chamadas são registradas no sistema.
	\item Usuários abre janela inicial com interface de acesso 
	\item Ingressar informações de acesso (usuário, senha) 
	\item Verificar aluno cadastrado 
	\item Verificar senha cadastrada 
	\item Liberar perfil de alunos 
	\item Liberar área e ferramentas de trabalho 
	\item Verifica quantidade máxima de chamadas por dia para não haver reclamações de atendimento.
	\end{itemize}


   \end{itemize}

   \subsection{Requisitos de Rede}
     \begin{itemize}
      \item O sistema realizará a automação dos processos seletivos academicos e permitirá que todo meio academico possa está sintonizado a suas notas e matrícuas.
      \item Deverá ser usada a programação Orientada a Objeto, pois atenderá melhor ao sistema.
      \item O usuário terá uma interface gráfica amigável para poder acessar o sistema.
      \item O sistema conterá diversas linguas, pois a universidade recebe alunos extrangeiros.
      \item A linguagem a ser utilizada deve ser portável.
     \end{itemize}
     
     \subsection{Requisitos de Subsistema}
      \begin{itemize}
       \item  O sistema deve ficar disponível 24 horas por dias, 7 dias por semana.
       \item  O sistema deverá utilizar dos melhores switchs, roteadores e cabos de redes de melhor qualidade, afim de não haver desgaste ou danos aos componetes.
       \item  A transmissão de dados deverá ser de 100mb/s para que o servidor não fique lento.
       \item O Sistema terá um sistema de backup do tipo RAD 10.
       \item É também de suma importância que cada subsistema tenha seu próprio backup físico para garantir à segurança dos dados do servidor, diminuindo o máximo possível a perda dos registros por meios de danos físicos causados por fatores externos.
      \end{itemize}

     \subsection{Definições}
       \begin{itemize}
        \item Funcionais
           \begin{itemize}
            \item O sistema deverá conter computadores robustos, pois armazenará todo os dados em um cloud, próprio.
            \item O sistema deverá se comunicar com eficiência.
            \item O sistema deverá fazer o cadastro de todas a comunidade academica.
            \item O servidor deverá monitorar e enviar emails para o administrado em caso de falha.
	    \item Monitoramento das notas dos alunos por meio de boletim.
           \end{itemize}
           
         \item Não funcionais
           \begin{itemize}
            \item O sistema utilizará o sistema operacional Linux.
            \item Manutenção preventiva.
            \item Um sistema com acesso simultâneo.
            \item Sistema com designer inovador.
            \item Cada usuário terá o seu tipo de acesso.
           \end{itemize}

        \item Características não desejáveis
	 \begin{itemize}
	  \item Instabilidade do sistema
	  \item Os meio academico não utilizar o serviço.
	  \item Baixa qualidade do sistema.
	  \item Sistema indisponível.
	 \end{itemize}
	 
	 \item Do produto
	 \begin{itemize}
	  \item Resultado satisfatório
	  \item Automatização das funções
	  \item Diminuição do uso do papel
	  \item Confiança na utilização
	  \item Satisfação do meio academico.
	 \end{itemize} 
	 
	 \item Da organização
	   \begin{itemize}
	    \item Extinção do papel.
	    \item Rapidez no serviço.
	    \item Automatização das tarefas.
	    \item Eficiência do serviço.
	    \item Comprimento do prazo combinado
	   \end{itemize} 
	   
	 \item Segurança
	   \begin{itemize}
	    \item Sigilo dos dados
	    \item Criptografia das senhas
	    \item Acesso permitido a usuário cadastrado.
	    \item Estabilidade do site	    
	    \item Confiança do site
	   \end{itemize}
	   
       \end{itemize}
