% Prof. Dr. Ausberto S. Castro Vera
% UENF - CCT - LCMAT - Curso de Ci\^{e}ncia da Computa\c{c}\~{a}o
% Campos, RJ,  2015
% Disciplina: An\'{a}lise e Projeto de Sistemas
% Aluno:


\chapter{Etapa de Planejamento}


\section{Solicitação de Sistemas - Projeto de Pedido pela Internet}


    \textbf{Responsável pelo Projeto:} \hfill Rodolfo Peixoto, CEO da InfinitIdeaInnovations

    \textbf{Necessidade da empresa:} \hfill Este projeto foi criado com o objetivo de atingir todos os funcionários da universidade levando agilidade,
    flexibilidade e facilidade à todo colegiado.

     \textbf{Requisitos de Negócios:}

     Os alunos serão capazes:

     \begin{itemize}
      \item Visualizar Notas
      \item Solicitar documentos
      \item Visualizar as matérias
      \item Incluir matérias
      \item Excluir matérias
      \item Grade do curso
      \item Visualizar as horas cursadas
     \end{itemize}

     Os docentes serão capazes:
     \begin{itemize}
      \item Criar Grade
      \item Aprovar matérias dos alunos
      \item Inserir listas de exercícios
      \item Inserir notas
      \item Inserir presença dos alunos
     \end{itemize}


    \textbf{Valor Agregado:}

    Esperamos que a Universidade Soft tenha mais agilidade, flexibilidade e facilidade para todo o colégiado, será automatizado todas as tarefas
    da universidade levando a economia com a diminuição dos papéis e aumentando os espaços, já que será retirado todos os armários que armazenam
    os papéis, dando lugar a um servidor. OS alunos e docentes terão economia no tempo.

      \textbf{Questões Especiais ou Restrições:}
      Para diminuir o custo todo o sistema será desenvolvido com aplicações de código aberto.

    \section{Valores, Custos e Benefícios}
 
  A Tab. \ref{VaCuBe} mostra os valores das componentes com seus benefícios.
    \begin{table}[h]
    \centering
    \begin{tabular}{|l|l|}
    \hline
	  \textbf{Benefícios}                          &                          \\ \hline
	  Aumento das Matrículas                       & R\$1.000.000,00          \\ \hline
	  Melhora do Serviço                           & R\$60.000,00             \\ \hline
	  Redução dos custos{[}funcionário{]}          & R\$700.000,00             \\ \hline
	  \textbf{Total dos Benefícios:}               & \textbf{R\$1.760.000,00} \\ \hline
						      &                          \\ \hline
	  \textbf{Custo de desenvolvimento}            &                          \\ \hline
	  1 Servidor                                   & R\$50.000,00             \\ \hline
	  Trabalho de desenvolvimento                  & R\$1.200.000,00          \\ \hline
	  \textbf{Total dos custos de desenvolvimento} & \textbf{R\$1.250.000,00} \\ \hline
						      &                          \\ \hline
	  \textbf{Custo Operacional}                   &                          \\ \hline
	  Hardware                                     & R\$54.000,00             \\ \hline
	  Trabalho operacional                         & R\$35.000,00             \\ \hline
	  \textbf{Total dos custos operacionais}       & \textbf{R\$89.000,00}   \\ \hline
	  \textbf{Total dos custos}       & \textbf{R\$3.099.000,00}  \\ \hline

    \end{tabular}
    \caption{Valores, Custos e Benefícios} \label{VaCuBe}
\end{table}


\newpage

    \section{Análise de Custo-Benefício}

 A Tab. \ref{CusBen} mostra todas as informações do custo benefício de cada componente.
\begin{table}[h]
\centering
 \begin{tabular}{|l|l|l|l|}
 \hline
                           & \scriptsize{\textbf{2014/1} } & \scriptsize{\textbf{2014/2} } & \scriptsize{\textbf{2015} } \\ \hline
 \scriptsize Aumento das Matrículas               & \scriptsize{R\$1.000.000,00}           &  \scriptsize{R\$1.000.000,00}    & \scriptsize{R\$1.000.000,00}    \\ \hline
  \scriptsize Melhora do Serviço                   & \scriptsize{R\$60.000,00}              &  \scriptsize{R\$60.000,00}    & \scriptsize{R\$60.000,00}   \\ \hline
  \scriptsize Redução dos custos{[}funcionário{]}  & \scriptsize{R\$700.000,00}             &  \scriptsize{R\$700.000,00}    & \scriptsize{R\$700.000,00}   \\ \hline
  \scriptsize{\textbf{Total dos Benefícios:} }     & \scriptsize{\textbf{R\$1.760.000,00}}  &  \scriptsize{\textbf{R\$1.760.000,00}}    & \scriptsize{\textbf{R\$1.760.000,00}}    \\ \hline
			                               &                                        &                       &    \\ \hline
  \scriptsize{\textbf{Custo de desenvolvimento}}   &               &   &       \\ \hline
	  \scriptsize 1 Servidor                 & \scriptsize{R\$50.000,00}              &   \scriptsize{R\$0}   & \scriptsize{R\$0}    \\ \hline
	  \scriptsize Trabalho de desenvolvimento          & \scriptsize{R\$1.200.000,00}           &   \scriptsize{R\$0}   & \scriptsize{R\$0}     \\ \hline
	  \scriptsize{\textbf{Total dos custos de desenvolvimento} } & \scriptsize{\textbf{R\$1.250.000,00}}  &   \scriptsize{R\$0}  & \scriptsize{R\$0}    \\ \hline
						                               &                                        &                       &  \\ \hline
	  \scriptsize{\textbf{Custo Operacional}}          &                                        &   \scriptsize{R\$0}   & \scriptsize{R\$0}  \\ \hline
	  \scriptsize Hardware                             & \scriptsize{R\$54.000,00}              &   \scriptsize{R\$0}   & \scriptsize{R\$0}   \\ \hline
	  \scriptsize Trabalho operacional                 & \scriptsize{R\$35.000,00}              &   \scriptsize{R\$35.000,00}  & \scriptsize{R\$35.000,00}   \\ \hline
	  \scriptsize{\textbf{Total dos custos operacionais} }  & \scriptsize{\textbf{R\$89.000,00} } &   \scriptsize{R\$35.000,00} & \scriptsize{R\$35.000,00}    \\ \hline
	  \scriptsize{\textbf{Total dos custos}}          & \scriptsize{\textbf{R\$3.099.000,00}}  &   \scriptsize{\textbf{R\$1.795.000,00}}   & \scriptsize{\textbf{R\$1.795.000,00}} \\ \hline
	  						                               &                                        &                       &  \\ \hline
	  \scriptsize{\textbf{Retorno do Investimento:}}          & \scriptsize{\textbf{70\%}}  &       &   \\ \hline
          \scriptsize{\textbf{Ponto de Equilíbrio:}}              &  \scriptsize{\textbf{6 meses}}  &       &   \\ \hline
          \scriptsize{\textbf{Benefícios Intangíveis:}}              &  \scriptsize{Automatização de serviços}  &       &   \\ \hline
    \end{tabular}
    \caption{Custo e Benefício do Projeto} \label{CusBen}
\end{table}



\newpage

    \section{Estudo de Viabilidade}
         A universidade deve construir um plano de negócio, organizando todas
         as informações coletadas sobre a mesma. O plano de negócio proporciona
         uma previsão do futuro da empresa e lhe prepara para o lucro ou prejuízo.

       \subsection{Viabilidade Técnica}

O serviço pela Internet é tecnicamente viável, embora haja alguns riscos:
 \begin{itemize}
  \item Alunos sem experiência na Internet
  \item Funcionários com dificuldades na utilização do sistema
  \item  Desvio de informações por meio de usuários mal intencionados
  \item  É preciso um DELL com 1TB, 16GB RAM e processador I7.
  \item A DELL presta suporte na empresa para o seus servidores
 \end{itemize}

 Porém é grandemente viável em vista da comodidade tanto para o
aluno quanto para todos da comunidade acadêmica e o baixo custo para manter um site em relação
a espaço de armazenamento de papéis.


 \subsection{Viabilidade Econômica}

Analisar a viabilidade econômica-financeira do projeto significa estimar e
analisar as perspectivas de desempenho financeiro do produto resultante do
projeto. Essa análise é iniciada na fase de Planejamento Estratégico do
Produto, pois ao escolher um dos produtos para ser desenvolvido, acredita-se
na viabilidade econômica-financeira de seu projeto. Nesse caso o produto final
oferecido é um sistema que possibilite o meio acadêmio agilidade na entrega de documentos,
visutalizar e inserir as notas dos alunos e outras tarefas que eram feitas manualmente.

\begin{flushleft}
Custos e Benefícios Intangíveis:
\end{flushleft}
 \begin{itemize}
  \item Melhora na satisfação dos usuários do meio acadêmico
  \item Maior reconhecimento da universidade, pois passa a imagem de uma instituição moderna
 \end{itemize}

 \begin{flushleft}
Custos e Benefícios Tangíveis:
\end{flushleft}
 \begin{itemize}
  \item Aumento das matrículas
  \item Diminuição dos funcionário
 \end{itemize}

  \subsection{Viabilidade Organizacional}

Sob uma perspectiva organizacional, esse projeto possui um risco baixo.
O objetivo do sistema, é agilizar e melhorar a usabilidade dos docentes e
alunos, portabilidade das informações e menor espaço físico para armazenar os documentos.

Espera-se que o meio acadêmico use os novos benefícios oferecidos
pela universidade. Além de ser uma forma mais rápida de realização de tarefas do meio acadêmico,
trazendo agilidade a todos os setores da universidade.


   \subsection{ Estudo de Viabilidade do Sistema }
   \begin{itemize}
    \item Cronograma
   \end{itemize}

 A Tab. \ref{Crono} trás os dados do início até o final(cronograma das atividades).

\begin{table}[h]
\centering
\begin{tabular}{|p{1.95cm}|p{1.8cm}|p{1.6cm}|p{1.7cm}|p{1.35cm}|p{1.25cm}|p{1.25cm}|p{1.3cm}|}
\hline
 \tiny{\textbf{Atividades}}               &  \tiny{\textbf{Dezembro/2014}}  &  \tiny{\textbf{Janeiro/2015}} &  \tiny{\textbf{Fevereiro/2015}} &  \tiny{\textbf{Março/2015}} &  \tiny{\textbf{Abril/2015}} &  \tiny{\textbf{Maio/2015}} &  \tiny{\textbf{Junho/2015}} \\ \hline
 \tiny{\textbf{Início do Projeto}}        &    \cellcolor{blue!25}            &              &                &            &            &            &            \\ \hline
 \tiny{\textbf{Análise do Projeto}}       &    \cellcolor{blue!25}            &              &                &            &            &            &            \\ \hline
 \tiny{\textbf{Compra dos Equipamentos}}  &               &   \cellcolor{blue!25}           &                &            &            &            &            \\ \hline
 \tiny{\textbf{Criação do Sistema}}       &               &   \cellcolor{blue!25}           & \cellcolor{blue!25}               & \cellcolor{blue!25}           &   \cellcolor{blue!25}         &            &            \\ \hline
 \tiny{\textbf{Teste do Sistema}}         &               &              &                &            &            &    \cellcolor{blue!25}        &            \\ \hline
 \tiny{\textbf{Treinamento dos usuários}} &               &              &                &            &            &            & \cellcolor{blue!25}           \\ \hline
\end{tabular}
\caption{Cronograma do Projeto} \label{Crono}
\end{table}


\begin{itemize}
 \item Calendário
\end{itemize}


A Tab. \ref{Calen} mostra o calendário com as datas do projeto.

\begin{table}[h]
\centering
\begin{tabular}{lll}
\hline
\multicolumn{3}{|c|}{Calendário}                                                                                    \\ \hline
\multicolumn{1}{|l|}{Atividades}               & \multicolumn{1}{l|}{Início}      & \multicolumn{1}{l|}{Termino}    \\ \hline
\multicolumn{1}{|l|}{Início do Projeto}        & \multicolumn{1}{l|}{10/12/2014}  & \multicolumn{1}{l|}{18/12/2014} \\ \hline
\multicolumn{1}{|l|}{Análise do Projeto}       & \multicolumn{1}{l|}{18/12/2015}  & \multicolumn{1}{l|}{01/01/2015} \\ \hline
\multicolumn{1}{|l|}{Compra dos Equipamentos}  & \multicolumn{1}{l|}{01//01/2015} & \multicolumn{1}{l|}{15/01/2015} \\ \hline
\multicolumn{1}{|l|}{Criação do Sistema}       & \multicolumn{1}{l|}{15/01/2015}  & \multicolumn{1}{l|}{04/04/2015} \\ \hline
\multicolumn{1}{|l|}{Teste do Sistema}         & \multicolumn{1}{l|}{05/05/2015}  & \multicolumn{1}{l|}{30/05/2015} \\ \hline
\multicolumn{1}{|l|}{Treinamento dos usuários} & \multicolumn{1}{l|}{04/06/2015}       & \multicolumn{1}{l|}{06/06/2015} \\ \hline
                                               &                                  &
\end{tabular}
\caption{Calendário com as datas do projeto} \label{Calen}
\end{table}

\newpage
\begin{itemize}
\centering
 \item Orçamento
\end{itemize}
O orçamento da universidade,como é mostrada na Tab. \ref{Orça} é necessário para que a empresa possa visualizar,
os custos envolvidos durante todo o projeto, havendo alternativas para que se
possa diminuir gastos e levar em conta a qualidade do produto fornecido.

\begin{table}[h]
\begin{tabular}{|p{4cm}|p{3cm}|p{3cm}|p{5cm}|p{2cm}|}
\hline
\multicolumn{5}{|c|}{\textbf{Orçamento}}                                                                                               \\ \hline
Cliente:                    & \multicolumn{4}{c|}{Universidade}                                                                        \\ \hline
Serviço:                    & \multicolumn{4}{c|}{Sistema Acadêmico}                                                                   \\ \hline
Data do pedido:             & \multicolumn{4}{c|}{10/12/2014}                                                                          \\ \hline
\textbf{Materiais direto}   & \textbf{Quantidade}                & \textbf{Custo Unitário} & \multicolumn{2}{c|}{\textbf{Custo Total}} \\ \hline
Servidor                    & 1                                  & R\$12.000               & \multicolumn{2}{c|}{R\$12.000}            \\ \hline
Roteador                    & 1                                  & R\$800,00               & \multicolumn{2}{c|}{R\$800,00}            \\ \hline
\multicolumn{3}{|c|}{\textbf{Total:}}                                                      & \multicolumn{2}{c|}{\textbf{R\$12800,00}} \\ \hline
\textbf{Custo Indireto}     & \textbf{Tempo( acerto + produção)} & \multicolumn{3}{c|}{\textbf{Custo Total}}                           \\ \hline
Internet                    & 1:00h+ 2:00h                       & \multicolumn{3}{c|}{R\$500,00/mês}                                  \\ \hline
Treinamento dos Funcionário & 0:30h + 48:00h                     & \multicolumn{3}{c|}{R\$50.000,00}                                   \\ \hline
Montagem do servidor        & 0:30h + 2:00h                      & \multicolumn{3}{c|}{R\$1600,00}                                     \\ \hline
\end{tabular}
\caption{Orçamento do Projeto}  \label{Orça}
\end{table}


\begin{itemize}
 \item Alternativa Tecnologica
\end{itemize}
Não há, pois foi utilizado as melhores tecnologias e com o melhor preço.

\begin{itemize}
 \item Recomendações
\end{itemize}
 Recomendamos que siga esse orçamento, pois mais inferior pode ocasionar em perda de qualidade do software.

